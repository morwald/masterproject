%-------------------------------------------------------------------------------
% GOVERNING EQUATIONS
%-------------------------------------------------------------------------------

\section{Governing Equations} \label{sec:gov_equ}

The following presents an overview of the physical and mathematical framework
for the 2D cavity flow problem introduced in section \ref{sec:driven_cav}. The
aim is to introduce the theory of the Navier-Stokes equations and the
stream-function formulation. \\

An incompressible Newtonian fluid in a domain $\Omega$ is governed by the
Navier-Stokes equations which are,
\vspace{-10pt}
\begin{align}
\frac{\partial \mathbf{u}}{\partial t} + 
  \mathbf{u} \cdot \nabla \mathbf{u} &= 
  - \frac{1}{\rho} \nabla p + \mu \nabla^2 \mathbf{u} + \mathbf{g},
  \label{eq:ns3d} \\
\nabla \cdot \mathbf{u} &= 0 \label{eq:cont3d},
\end{align}

where $\mathbf{u}$ denotes the velocity vector (2D or 3D). $p$ is the pressure,
$\rho$ is the constant fluid density, $\nu$ is the kinematic viscosity and
$\mathbf{g}$ defines the body acceleration acting on the fluid. The first
equation represents the conservation of momentum, while the second is the
continuity equation. Furthermore, boundary conditions must be imposed on the
domain $\Omega$ for these equations. \\

For cavity flow problems in a plane, we consider the 2D form of the above
equations without body acceleration. They can be written explicitly for the two
spatial components as,
\begin{align}
\frac{\partial u}{\partial t} + u \frac{\partial u}{\partial x} 
  + v \frac{\partial u}{\partial y} &= 
  - \frac{1}{\rho}\frac{\partial p}{\partial x}
  + \nu \left(\frac{\partial^2 u}{\partial x^2}
  + \frac{\partial^2 u}{\partial y^2}\right) \label{eq:ns2d-u}, \\
\frac{\partial v}{\partial t} + u \frac{\partial v}{\partial x}
  + v \frac{\partial v}{\partial y} &=
  - \frac{1}{\rho}\frac{\partial p}{\partial y} 
  + \nu \left(\frac{\partial^2 v}{\partial x^2}
  + \frac{\partial^2 v}{\partial y^2}\right) \label{eq:ns2d-v}, \\ 
\frac{\partial u}{\partial x}
  + \frac{\partial v}{\partial y} &= 0 \label{eq:cont2d}.
\end{align}

Here, $u$ and $v$ denote the velocities in the $x$ and $y$ directions,
respectively. Equations \eqref{eq:ns2d-u} to \eqref{eq:cont2d} build the
underlying framework for the analysis of the 2D cavity flow.

\clearpage

In the case of a 2D incompressible fluid, it is possible to introduce a scalar
function $\Psi(x,y,t)$ called the streamfunction which is defined such that,
\begin{align}
u & = \frac{\partial \Psi}{\partial y}, \label{eq:str_defx} \\
v & = -\frac{\partial \Psi}{\partial x}. \label{eq:str_defy} 
\end{align}

By its definition, the streamfunction satisfies the continuity equation and
therefore the incompressibility condition. Regarding the momentum equations,
the expressions \eqref{eq:str_defx} and \eqref{eq:str_defy} can be used to
obtain a formulation of the Navier-Stokes where the pressure can be eliminated
and thus only involves the streamfunction and \citep{landau1987}:
\begin{align}
\partial_t \Delta \Psi = \nu \Delta^2 \Psi
  + (\partial_x \Psi) \partial_y(\Delta \Psi)
  - (\partial_y \Psi) \partial_x(\Delta \Psi). \label{eq:str_dim}
\end{align}

For a shorter notation, the partial derivates of the streamfunction are denoted
as subscripts in the equation above and from now on. It is important to note
that $\Psi = constant$ represents the family of curves of the streamline
\citep{landau1987}. Hence, if we know the streamfunction, we can visualize the
streamlines by setting the function to different constant values. \\

Viewing a non-dimensional version of the problem is standard for analyzing the
cavity flow (figure \ref{fig:cav_4s}). The relations \eqref{eq:scl} show the
characteristic scales of the 2D cavity. We will use the same length scale $l$
for both sides, i.e., a square and thus an aspect ratio of $1$. Furthermore, as
the magnitude for our regularized boundary conditions is set to be the equal
for all lids, we can use the same velocity scale $U$.
\vspace{-5pt}
\begin{equation}
\begin{split}
\left[ l \right] &= length  \\
\left[ U \right] &= length*time^{-1} \\
\left[ \Psi \right] &= length^2*time^{-1} \\
\left[ \frac{l}{U} \right] &= time \quad \text{(dynamic time)} \\
\end{split}
\label{eq:scl}
\end{equation}

All parameters can now be made dimensionless by defining $x = l x^*$, $y = l
y^*$, $t = \frac{l}{U} t^*$, $\Psi = lU \Psi^*$. Additionally, the
non-dimensional operators are scaled as $\partial_t = \frac{U}{l}
\partial_{t^*}$, $\partial_x = \frac{1}{l} \partial_{x^*}$, $\partial_y =
\frac{1}{l} \partial_{y^*}$ and $\Delta_* = \frac{1}{l^2} \Delta_{x^*}$. By
plugin these definitions into \eqref{eq:str}, simplifying and dividing by
$\frac{U^2}{L^2}$, we get,
\begin{align}
\partial_{t^*} \Delta_* \Psi^* = \frac{\nu U}{l} \Delta^2_* \Psi^*
  + (\partial_x^* \Psi^*) \partial_y^*(\Delta_* \Psi^*)
  - (\partial_y^* \Psi^*) \partial_x^*(\Delta_* \Psi^*). 
\end{align}

We notice that it is possible to recover the Reynolds number, $\Rey =
\frac{Ul}{\nu}$ that, as a non-dimensional parameter, characterizes the
relative importance of inertial forces and viscous forces a flow. For clarity,
we will omit $*$ notation. All quantities will correspond to the dimensionless
variables. The final equation reads,
\vspace{-5pt}
\begin{align}
\partial_t \Delta \Psi = \frac{1}{\Rey} \Delta^2 \Psi
  + (\partial_x \Psi) \partial_y(\Delta \Psi)
  - (\partial_y \Psi) \partial_x(\Delta \Psi). \label{eq:str}
\end{align}

This equation \eqref{eq:str} will be the foundation for the numerical
investigation later on. The equation is non-dimensional and only depends on the
scalar-valued streamfunction and the Reynolds number. It is a nonlinear
ordinary differential equation of order 4, where we can analyze different
dynamics and regimes by changing the Reynolds number.
