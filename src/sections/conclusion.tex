%-------------------------------------------------------------------------------
% CONCLUSION AND FUTURE WORK
%-------------------------------------------------------------------------------

\section{Conclusion and Future Work} \label{concl}

We have seen that the problem involves more bifurcation scenarios than
anticipated. We want to finish with the discussion on further improvements and
future work. The abovementioned more complicated crossing at the saddle-node
needs further refinement of the states of the continuation algorithm. An exact
comparison to the MATLAB code at a higher resolution of the points on the
continuation curve would be required to clarify the issue. Nevertheless, it
seems that the Hopf bifurcation and pitchfork $P_3$ can be seen independent of
the problem arising at the saddle-node. 

Another possible future investigation is related to the symmetries of the
cavity. Using a square cavity and having the four lid's velocities changing at
the same time means that we are varying 4 Reynolds numbers simultaneously. From
a geometrical point of view, this indicates the movement along a line in a
four-dimensional space, where the four Reynolds numbers could be varied. To get
a better picture of what is happening, one could break the symmetries of the
problem and impose different velocities at the four lids and, thus, different
Reynolds numbers. This could separate the dense scenario of eigenvalues near
the saddle-node.

Also, in future research, it has to be clarified how the stable upper branch is
separated from the stable orbits below the critical Reynolds number of the
Hopf. Moreover, we have always worked with the regularized version to be sure
of the convergence rate for our pseudo-spectral discretization. It could be
interesting to see if one can reproduce the periodic orbits and the secondary
branch and compare the critical Reynolds numbers in the discontinuous case.
Conversely, \citet{chen2013} used a finite difference discretization. With the
pseudo-spectral approach, it is certainly not guaranteed to get a useful
approximation of the original boundary conditions. Another strategy is trying
the spectral approach on the non-regularized version and seeing how the
convergence rates and results are affected.

Lastly, it has to be mentioned that there is another eigenvalue crossing
happening close to the pitchfork $P_2$ of the unstable base solution. This
phenomenon was not investigated in this study but could suggest a more complex
picture of the bifurcation scenario at $P_2$.

In terms of implementation, the Julia module could be further extended. As a
first step, the code can be adapted for rectangular grids. Also, the
continuation algorithm and the linear stability analysis results are saved
using a CSV format, and the values of the streamfunctions at the grid points
are stored in plain text files. A further optimization could be to save results
as binaries in an HDF5 format directly. This could easily be done with a
package called \emph{JLD.jl} and would probably reduce the code size and
simplify the results handling. Another improvement worth mentioning is that the
code is divided into the core functions explained in section \ref{sec:impl} and
into the code necessary to run the simulations and generate the plots. The
Julia module only includes the core functions mentioned in the implementation
section for readability. It could be beneficial to have both parts reconciled,
and the binary savings of results would greatly help. \\

In conclusion, this master project focused on the four-sided lid-driven cavity
flow using regularized boundary conditions. The study's spectral Chebyshev
discretization technique combined with the regularization yields highly
accurate results having an exponential convergence rate as the grid size
increases. We have seen that the R4CF exhibits a wide variety of bifurcation
scenarios at much more moderate Reynolds numbers so that different
discretizations of the Navier-Stokes equations can be easily tested.

The regularized version recovers the asymmetric branches of the problem with
discontinuous boundary conditions. The original bifurcations have been
detected, namely the two pitchforks of the base flow and the saddle-node of the
non-regularized version. Further, in the regularized version, a Hopf
bifurcation of the asymmetric solution branch was found at a Reynolds number of
$348.321$. Additionally, a third pitchfork occurs close to the saddle-node at
$353.358$ in Reynolds. The connected unstable branch breaks the $\pi$-rotation
symmetry of the asymmetric solutions. On the other hand, the saddle-node
bifurcation needs further investigation due to the unclear eigenvalue crossing,
possibly associated with a complex conjugate pair. The presence of three
pitchfork bifurcations and the Hopf bifurcation in this problem offers robust
scenarios to benchmark a Navier-Stokes solver. The regularized four-sided
lid-driven cavity only requires slight adaptations from the classical one-sided
version, making it easily applicable to existing codes. This modified setup
allows for testing the multiplicity of states and periodic orbits at low to
moderate Reynolds numbers providing the means to assess a solver's capability
in identifying these solution types and the oscillatory behavior. In general,
it can make up for an interesting bifurcation benchmark.
