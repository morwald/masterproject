%-------------------------------------------------------------------------------
% BIFURCATION ANALYSIS 
%-------------------------------------------------------------------------------

\section{Bifurcation Analysis and Dynamical Systems} \label{sec:bif}

Hydrodynamic stability theory studies how a fluid flow is affected by small
disturbances of an initial state. This field of fluid dynamics involves
analytical, experimental, and more and more computational explorations. Flow
regimes can be categorized as either unstable or stable. Unstable in the sense
that even infinitesimally small variations will cause the flow to deviate from
its initial state, resulting in a different flow state or the onset of
turbulence. Conversely, disturbances do not change the initial system's state in
a stable flow.

In the context of the R4CF problem, the main objective is to investigate the
behavior of the driven flow for different configurations in a numerical
experiment. We want to mainly understand how the flow is affected by the length
of the side-walls, the boundary velocity, and the fluid's viscosity. Moreover, all
these quantities are captured within the non-dimensional parameter, the
Reynolds number.

From the preliminary results shown in the temporal discretization section
\ref{sec:time}, it has been identified that two asymmetric states are possible
after a critical Reynolds number, and although the symmetric base flow remains
a solution to the governing equations, it becomes unstable. The point at which
the two states emerge is called a bifurcation point. A bifurcation is the
change of a value of qualitative character and a set of possible steady and
unsteady flows in dynamical equilibrium (\ref{drazin}). These points are often
associated with instabilities, multiple flow patterns, or the emergence of
oscillatory behavior. Recalling the \ref{bif_diag} it is aimed to distinguish
the different points where the flow states are altered. 

To understand this bifurcation diagram, we want to introduce the equations from
a dynamical system point of view. Thenafter, the occurring bifurcations for the
4RCF will be explained in detail.

\subsection{The Dynamical Systems Point of View}

An alternative perspective on our problem is through the theory of dynamical
systems. In this framework, the state of a system is represented by a point
denoted as $x$ within a set of all possible states, referred to as the state
space or phase space (denoted as $X$). This point not only describes the
system's current position but also includes the necessary dimensions needed to
determine its future evolution.

A time variable $t \in T$ governs the evolution of a system, where $T$ is a set
of positive real numbers ($T \in \mathbb{R}^+$). The evolution operator $\varphi_t
: X \to X$ is a map that describes the system's transformation over time. For
the evolution to comply with the deterministic nature of dynamical systems, the
evolution operator must satisfy two conditions defined in the book of
\cite{kuznetsov2004}:

\begin{align}
  \varphi^0 &= id, \label{eq:op_cond1}, \\
  \varphi^{t + s} &= \varphi^t \circ \varphi^s. \label{eq:op_cond2}
\end{align}

The first condition corresponds to the identity map where for an initial
condition $x_0$ at time $0$, the map should result in $x_0$ again. The second
statement is related to time invariance and signifies that for any $x \in X$,
the evolution operator $\varphi_t$ remains constant over time, that is, $\varphi_t(x)
= \varphi_s}(x)$ for all $t, s \in T$. \\

Now, a dynamical system can be defined as follows:

\begin{definition}
A dynamical system is a triple $ \{ T, X, \varphi^t \} $, where $T$ is a time set,
$X$ is a state space, and $\varphi^t : X \to X$ is a family of evolution operators
parametrized by $t \in T$ and satisfying the properties \eqref{eq:op_cond1} and
\eqref{eq:op_cond2} \citep{kuznetsov2004}.
\end{definition}

The equation for the streamfunction \eqref{eq:str} is an autonomous
differential equation, which can be reformulated as a dynamical system. We
consider an infinite-dimensional point denoted as $\Psi$ in a state space $X$.
Using the streamfunction at time $t$, we not only have the locations of all
fluid particles but as well their evolution (velocities $u$ and $v$) explicitly
given by the definition of the derivatives \eqref{eq:str_def}.

This infinite-dimensional space is divided into discrete points with the chosen
pseudo-spectral discretization. As already mentioned, theory of spectral
discretization ensures that this approximation converges to the mathematical
solution of the infinite-dimensional state space. Increasing $N$ achieves
higher accuracy and is expected to converge exponentially because of to the
regularization. The finite-dimensional space for the numerical computations can
be defined as:

\begin{align}
  \frac{d(\Delta \Psi_j)}{dt} &= {F_j(\Psi)} \quad \quad \quad
    j= 1,2, ...,(M+1)(N+1) \label{eq:dyn_sys}
\end{align}

where $F_j$ to the $j$th component of of the nonlinear function of a grid point.
Equations \eqref{eq:dyn_sys} is essentially a dynamical system of $(M+1)(N+1)$
equations.

This more abstract framework is helpful in the sense that it provides a
geometrical representation of the sought-after solutions. 
What follows are some useful definitions:

\begin{definition}
An invariant set of a dynamical system $\{T, X, \varphi^t\}$ is a subset set such
that $x_0$ that implies $\varphi(x_0) \in S$ for all $t \in T$. \citep{kuznetsov2004}. 
\end{definition}

Additionally, what is referred to as an invariant manifold corresponds to a
hypersurface such that all future times maps of the evolution operator reside
on this surface.

Another important concept are fixed points (or equilibria). This points where
$\varphi^t(x_0) = x_0$ for all time $t \in T$. 

\todo{Introduce periodic orbits limit cycles}


\subsection{Steady-State Solutions}

We have seen that launching a time stepper gives us different kinds of states
depending on the initial condition. A steady state solution corresponds to the
solution of the differential equation where the time $t$ approaches infinity.
Moreover, this is exactly a fixed point (and invariant set) that can be
obtained by setting the time-dependent term to zero. All steady-state solutions
satisfy the following:

\begin{align}
  F(\Psi, \Rey) & = \frac{1}{\Rey}\Delta^2 \Psi +
    (\partial_x \Psi) \partial_y(\Delta \Psi) -
    (\partial_y \Psi) \partial_x(\Delta \Psi) \nonumber \\
  & =  0 \label{eq:str_steady}
\end{align}

As the outer grid points are explicitly known through the boundary conditions
\ref{sec:bc}, a reduced system of equations $F(\psi, \Rey) = 0$ can be
formulated. $\psi \in \mathbb{R}^{(m-1)\times(n-1)} $ corresponds now to the
inner grid points. 

Practically in the case of the two-dimensional cavity flow, it will have a
system of $(M-3) \cdot (N-3)$ equations. Which result

\subsection{Linearstability Analysis}

To analyze the stability of fixed points, a mathematical technique called
linear stability analysis can be used. Given an equilibrium solution $\Psi_0$
which is a solution to the steady equation \ref{eq:str_steady}, we can try to
perturb the system and see how it would behave. Let us define a perturbated
$\Psi$,

\begin{align}
\Psi = \Psi_0 + \epsilon \tilde{\Psi},
\end{align}

where $\epsilon$ is a small disturbance. We can insert this expression into the
time-dependent streamfunction equation. By only looking at first-order terms
$\mathcal{O}(\epsilon)$ and using the fact that $F_{steady}(\Psi_0) = 0$, we
get

\begin{align}
\partial_t \Delta \tilde{\Psi} = \frac{1}{Re} \Delta^2 \tilde{\Psi}
  + (\partial_x \Psi_0) \partial_y (\Delta \tilde{\Psi})
  + (\partial_x \tilde{\Psi}) \partial_y (\Delta \Psi_0)
  - (\partial_y \Psi_0) \partial_x (\Delta \tilde{\Psi})
  - (\partial_y \tilde{\Psi}) \partial_x (\Delta \Psi_0)
\label{eq:str_pert}
\end{align}

To solve this linear stability analysis problem, an Ansatz is thought for,
where we separate time and the spatial variables as follows: 

\begin{align}
  \tilde{\Psi} = \tilde{\Psi}(x,y,t) = \mathrm{e}^{\lambda t} \Phi(x,y)
\end{align}

$\Phi$ only corresponds to a component of the solution which depends on $x$ and
$y$. The time variable $t$ can be eliminated by plugin the above back into the
equation.

\begin{align}
\lambda \Delta \Phi = \frac{1}{Re} \Delta^2 \Phi
  + (\partial_x \Psi_0) \partial_y (\Delta \Phi)
  + (\partial_x \Phi) \partial_y (\Delta \Psi_0)
  - (\partial_y \Psi_0) \partial_x (\Delta \Phi)
  - (\partial_y \Phi) \partial_x (\Delta \Psi_0)
\label{eq:str_phi}
\end{align}

This equation is a generalized eigenvalue problem. In terms of two operators
$A$ and $B$, the problem can be stated as follows: 

\begin{align} \label{eq:eig_prob}
  \begin{split}
  A \Phi & = \lambda B \Phi \\[4pt]
  \begin{split}
  A & = \frac{1}{Re} \Delta^2 \bullet
    + (\partial_x \Psi_0) \partial_y (\Delta \bullet)
    + (\partial_x \bullet) \partial_y (\Delta \Psi_0) \\
    &\quad - (\partial_y \Psi_0) \partial_x (\Delta \bullet)
    - (\partial_y \Phi) \partial_x (\Delta \Psi_0)
  \end{split} \\
  B & = \Delta
  \end{split}
\end{align}

We recognize that $\Phi(x,y)$ corresponds to an eigenmode, and $\lambda$ is the
associated eigenvector. If all the eigenvalues of the generalized eigenvalue
problem have a positive real part, the equilibrium is stable. On the contrary,
if only one of the eigenvalues is found to be positive, then the system is
unstable because a small perturbation will grow exponentially in time which can
be directly seen in the definition of the Ansatz.

This linear stability problem becomes very large because the perturbation
$\eps$ has to be performed on every grid point, and the numerical eigenvalue
problem is solved with matrices of the size $(M+1)(N+1) \times (M+1)(N+1)$.

\subsection{Bifurcations}

Now we have formally defined all the tools need to precisly characterize all
the bifurcations occuring in the RCF4. To study equation \eqref{dyn_sys}, the
equation we now vary the $\Rey$ number. As we deal with only partial
differential equation our problem can be approached from a dynamical system
point of view as described before. To illustrate such bifurcations normally a
characteristic point in the system and it's. Let us use $U$ to. The first
bifurcation we want to have a look at is the pitchfork \ref{fig:pitchfork}
where the symmetric is broken and we have two different flow solutions
appearing from. The base flow is still an equilibria (i.e. )  but now unstable.

\begin{figure}[ht]
\centering
  \begin{subfigure}[t]{0.3\textwidth}
    \centering
    % \scalebox{0.7}{
      \begin{tikzpicture}[scale=0.6]
        \draw[->] (-0.5,0) -- (6,0) node[right] {$\mu$};
        \draw[->] (0,-3) -- (0,3) node[above] {$U$};
        
        \draw[thick] plot[smooth,domain=-2:2] ({1 + \x*\x},\x);
        \draw[thick] plot[smooth,domain=0:1] (\x,0);
        \draw[thick] plot[smooth,domain=0:1] (\x,0);

        \draw[thick, dashed] plot[smooth,domain=1:5.2] (\x,0);
        \draw[thick, dashed] plot[smooth,domain=1:5.2] (\x,0);
        
        \draw[fill=custom] (1,0) circle (0.05);
        \node[above left] at (1,0) {$\mu_c$};
      \end{tikzpicture}
    % }
    \caption{Supercritical pitchfork bifurcation}
    \label{fig:pitchfork}
  \end{subfigure}
  \hspace{0.1\textwidth}
  \begin{subfigure}[t]{0.3\textwidth}
    \centering
    % \scalebox{1}{
      \begin{tikzpicture}[scale=0.6]

      \draw[->] (-0.5,0) -- (6,0) node[right] {$\mu$};
      \draw[->] (0,-3) -- (0,3) node[above] {$U$};

      \draw[thick, dashed] plot[smooth,domain=-1.9:0] ({5 - 1.2*\x*\x},\x);
      \draw[thick] plot[smooth,domain=0:1.9] ({5 - 1.2*\x*\x},\x);

      \draw[fill=custom] (5,0) circle (0.05);
      \node[above right] at (5,0) {$\mu_c$};

      \end{tikzpicture}
    % }
    \caption{Saddle node bifurcation}
    \label{fig:saddlenode}
  \end{subfigure}
\end{figure}

Another type of bifurcation is called a saddle node or fold bifurcation where a
set of stable and unstable fixed points collide. In a phase diagram in figure
\ref{fig:saddlenode} this is defined by a saddle point hence the name.

A Hopf bifurcation named after Eberhard Hopf is a bifurcation where a stable
fixed point looses its stability and from the critical we have unstable or
stable (limit cycles) periodic orbits. This is characterized by the real
complex conjugate pair of eigenvalues crossing when performing the stability
analysis. One can distinguish between sub- and super critical Hopf bifurcation
which are depicted in \ref{fig:hopf}. Compared to the other bifurcation this is
a global bifurcation as these behaviors such at. On the other hand pitchforks
and saddle node are called local bifurcations.

\subsection{Continuation Algorithms}

Having detected this different asymmetric solutions in the prelimary results
the next key idea is how to actually track such a branches of fixed points.  

The simplest way is called natural continuation where we first obtain
solution obtained by timestepping and further converged with the Newton solver
then increase the Reynolds number. The initial guess for the next Newton solve
is now the from befor. In this way we can "continue" a branch and don't
have to use timestepping algorithm for each steady-state solution we want to
obtain. Furthermore an unstable branch can be followed as well which would not
be able to be tracked solely by time evolution.

But there is actually a limitation. In the fold bifurcation depicted in figure
\ref{fig:saddlenode} the continuation has to follow a curve when the parameter
is decreasing again. This can't be detected by the natural continuation.
Another strategie to overcome this issue is called pseudo-arclength
continuation and was introduced by Keller in the 1970th \citep{}. We want to
use a tangent to approximate the curve. Given two fixed points $u^{(0)}$ and
$u^{(1)}$  which parameter values $\mu^{(0)}$ and $\mu^{(1)}$ we want to find
the next $u^{(0)}$ and $\mu^{(2)}$. First of let's make a prediction using the approximated
tangents from the already given fixed points:

\begin{equation}
  \tilde{u}^{(2)} = u^{(1)}  + \underbrace{[u^{(1)} - u^{(0)}]}_{\text{\normalfont $\hat{u}$}} \gamma
\end{equation}

Here $\gamma$ is a parameter which determines the prediction step size.
$\tilde{u}^{(2)}$ is called the predictor. Now we want to impose another
equation with what is called the corrector.

\begin{equation}
  (u^{(2)} - u^{(1)})  \cdot \hat{u} \label{eq:extra}
\end{equation}

This equation tells us that we want to find the next point on the curve such that it   
is in our phase space orthogonal to the tangent drawn from our previous points. Figure
\ref{fig:cont} illustrates this idea geometrically.

Consequently, $\mu^{(2)}$ is determined explicitly by adding the equation
\eqref{eq:extra}. To practicaly find the next point we want to augment our
initial system of nonlinear equations \eqref{eq:dyn_sys}:

\begin{equation}
  F(u^{(2)}, \mu^{(0)}) = 
\begin{bmatrix} F(u^{(2)}, \mu) \\ (u^{(2)} - u^{(1)})  \cdot \hat{u}
\end{bmatrix} \label{eq:dyn_sys_cont}
\end{equation}

This means the parameter $\mu$ is now part of the equation. The Jacobian can be
formulated as follows: 

\begin{center}
\[
\renewcommand\arraystretch{1.5}
J = 
\left[
\begin{array}{c:c}
  J_u \quad \quad \quad \quad & F_{\mu} \\
  \hdashline
  (u^{(2)} - u^{(1)})^T
\end{array} \label{eq:jac_aug}
\right]
\]
\end{center}

So we first of all solve the normal system with the jacabian and the last line
of our jacobian is given by the derivative of the scalar with respect to each
of the components. It is necessary to   scale the the change in $\lambda$ to
get a well-conditioned Jacobian.

\begin{figure}[ht]
\centering

\begin{subfigure}[t]{0.3\textwidth}
\centering
\begin{tikzpicture}
  \draw[->] (0,-0.5) -- (0,4) node[above] {$"u"$};
  \draw[->] (-0.5,0) -- (4,0) node[right] {$\lambda$};
  
  \draw[smooth, thick, variable=\x, domain=0.5:4] plot ({\x},{1.2 + 0.2*(\x - 1)^2});
  
  \draw[fill=black] (0.6,{1.2 + 0.2*(0.6 - 1)^2}) circle (0.03) node[below] {$u^{(0)}$};
  \draw[fill=black] (1.6,{1.2 + 0.2*(1.6 - 1)^2}) circle (0.03) node[below] {$u^{(1)}$};
  
  % % Tangent line at initial point
  \draw[dashed] (0.6,{1.2 + 0.2*(0.6 - 1)^2}) -- (1.6,{1.2 + 0.2*(1.6 - 1)^2});
  % 
  % % Displaced point
  % \draw[fill=red] (-0.2,0.5) circle (0.03) node[above] {$P$};
  % 
  % % Displacement vector
  % \draw[->, thick] (-1.2,-0.5) -- (-0.2,0.5) node[midway, above left] {$\Delta x$};
  % 
  % % Arc length marker
  % \draw[<->] (-0.8,-0.3) -- (-0.8,0.3) node[midway, left] {$\Delta s$};
\end{tikzpicture}

 \label{fig:saddlenode}
  \end{subfigure}
\end{figure}

Another type of bifurcation is called a saddle node or fold bifurcation where a
set of stable and unstable fixed points collide. In a phase diagram in figure
\ref{fig:saddlenode} this is defined by a saddle point hence the name.

A Hopf bifurcation named after Eberhard Hopf is a bifurcation where a stable
fixed point looses its stability and from the critical we have unstable or
stable (limit cycles) periodic orbits. This is characterized by the real
complex conjugate pair of eigenvalues crossing when performing the stability
analysis. One can distinguish between sub- and super critical Hopf bifurcation
which are depicted in \ref{fig:hopf}. Compared to the other bifurcation this is
a global bifurcation as these behaviors such at. On the other hand pitchforks
and saddle node are called local bifurcations.

\subsection{Continuation Algorithms}

Having detected this different asymmetric solutions in the prelimary results
the next key idea is how to actually track such a branches of fixed points.  

The simplest way is called natural continuation where we first obtain
solution obtained by timestepping and further converged with the Newton solver
then increase the Reynolds number. The initial guess for the next Newton solve
is now the from befor. In this way we can "continue" a branch and don't
have to use timestepping algorithm for each steady-state solution we want to
obtain. Furthermore an unstable branch can be followed as well which would not
be able to be tracked solely by time evolution.

But there is actually a limitation. In the fold bifurcation depicted in figure
\ref{fig:saddlenode} the continuation has to follow a curve when the parameter
is decreasing again. This can't be detected by the natural continuation.
Another strategie to overcome this issue is called pseudo-arclength
continuation and was introduced by Keller in the 1970th \citep{}. We want to
use a tangent to approximate the curve. Given two fixed points $u^{(0)}$ and
$u^{(1)}$  which parameter values $\mu^{(0)}$ and $\mu^{(1)}$ we want to find
the next $u^{(0)}$ and $\mu^{(2)}$. First of let's make a prediction using the approximated
tangents from the already given fixed points:

\begin{equation}
  \tilde{u}^{(2)} = u^{(1)}  + \underbrace{[u^{(1)} - u^{(0)}]}_{\text{\normalfont $\hat{u}$}} \gamma
\end{equation}

Here $\gamma$ is a parameter which determines the prediction step size.
$\tilde{u}^{(2)}$ is called the predictor. Now we want to impose another
equation with what is called the corrector.

\begin{equation}
  (u^{(2)} - u^{(1)})  \cdot \hat{u} \label{eq:extra}
\end{equation}

This equation tells us that we want to find the next point on the curve such that it   
is in our phase space orthogonal to the tangent drawn from our previous points. Figure
\ref{fig:cont} illustrates this idea geometrically.

Consequently, $\mu^{(2)}$ is determined explicitly by adding the equation
\eqref{eq:extra}. To practicaly find the next point we want to augment our
initial system of nonlinear equations \eqref{eq:dyn_sys}:

\begin{equation}
  F(u^{(2)}, \mu^{(0)}) = 
\begin{bmatrix} F(u^{(2)}, \mu) \\ (u^{(2)} - u^{(1)})  \cdot \hat{u}
\end{bmatrix} \label{eq:dyn_sys_cont}
\end{equation}

This means the parameter $\mu$ is now part of the equation. The Jacobian can be
formulated as follows: 

\begin{center}
\[
\renewcommand\arraystretch{1.5}
J = 
\left[
\begin{array}{c:c}
  J_u \quad \quad \quad \quad & F_{\mu} \\
  \hdashline
  (u^{(2)} - u^{(1)})^T
\end{array} \label{eq:jac_aug}
\right]
\]
\end{center}

So we first of all solve the normal system with the jacabian and the last line
of our jacobian is given by the derivative of the scalar with respect to each
of the components. It is necessary to   scale the the change in $\lambda$ to
get a well-conditioned Jacobian.

\begin{figure}[ht]
\centering

\begin{subfigure}[t]{0.3\textwidth}
  \centering
  \begin{tikzpicture}
    \draw[->] (0,-0.5) -- (0,4) node[above] {$"u"$};
    \draw[->] (-0.5,0) -- (4,0) node[right] {$\lambda$};
    
    \draw[smooth, thick, variable=\x, domain=0.5:4] plot ({\x},{1.2 + 0.2*(\x - 1)^2});
    
    \draw[fill=black] (0.6,{1.2 + 0.2*(0.6 - 1)^2}) circle (0.03) node[below] {$u^{(0)}$};
    \draw[fill=black] (1.6,{1.2 + 0.2*(1.6 - 1)^2}) circle (0.03) node[below] {$u^{(1)}$};
    
    \draw[dashed] (0.6,{1.2 + 0.2*(0.6 - 1)^2}) -- (1.6,{1.2 + 0.2*(1.6 - 1)^2});
  \end{tikzpicture}
  \caption{Natural continuation}
  \label{fig:nat_cont}
\end{subfigure}
\hspace{0.1\textwidth}
\begin{subfigure}[t]{0.3\textwidth}
  \centering
  \begin{tikzpicture}
    \draw[->] (0,-0.5) -- (0,4) node[above] {$"u"$};
    \draw[->] (-0.5,0) -- (4,0) node[right] {$\lambda$};
    
    \draw[smooth, thick, variable=\x, domain=0.5:4] plot ({\x},{1.2 + 0.2*(\x - 1)^2});
    
    \draw[fill=black] (0.6,{1.2 + 0.2*(0.6 - 1)^2}) circle (0.03) node[below] {$u^{(0)}$};
    \draw[fill=black] (1.6,{1.2 + 0.2*(1.6 - 1)^2}) circle (0.03) node[below] {$u^{(1)}$};
    
    \draw[dashed] (0.6,{1.2 + 0.2*(0.6 - 1)^2}) -- (1.6,{1.2 + 0.2*(1.6 - 1)^2});
  \end{tikzpicture}
  \caption{Pseudo-arclength continuation}
  \label{fig:nat_cont}
\end{subfigure}
\caption{Sketch of continuation algorithms, 
  $"u"$ refers to a projection onto a plane for illustration purposes}
\end{figure}
