%-------------------------------------------------------------------------------
% INTRODUCTION 
%-------------------------------------------------------------------------------

\section{Introduction} \label{sec:intro}

Driven cavity flows have long served as essential benchmarks for validating
Navier-Stokes solvers. These problems can test spatial discretization
methodologies such as finite elements, finite differences, and spectral
methods. They also assess a variety of boundary condition implementations and
time-stepping schemes.

Current theoretical and computational Navier-Stokes flows research goes beyond
the time integration of the equations of motion. Since the publication of the
seminal work by Eberhard Hopf 75 years ago \citep{hopf1948}, bifurcation and
dynamical systems theory have been vindicated as suitable deterministic
frameworks to understand hydrodynamic instabilities and the transition to fluid
flow turbulence.

Exploring the infinite-dimensional phase space of the Navier-Stokes equations
is necessary to anticipate instabilities and comprehend the emergence of new
flow states. That implies the seek for invariant sets or manifolds in that
space, the simplest cases being equilibria (steady flows) and periodic orbits
(time-periodic flows). These manifolds are often linearly unstable, although
dynamically relevant, and therefore systematically overlooked by time
integrators of the Navier-Stokes equations, as time-stepping will only approach
local attractors.

Accurately computing steady or time-periodic flows, regardless of their
stability requires the implementation of suitable continuation algorithms
\citep{kuznetsov2004} to monitor them as the Reynolds number is varied.
Moreover the linear stability analysis of these invariant flows is crucial for
predicting potential local bifurcations in parameter space. However, this
analysis poses challenges due to the high-dimensional generalized eigenvalue
problem resulting from the local linearization of the Navier-Stokes operator.

Numerically reliable Navier-Stokes benchmarks require high accuracy and
robustness of its solutions. It is well known that some spatial numerical
discretizations of the Navier-Stokes equations may lead to spurious solutions
that typically appear for coarse grids, eventually vanishing when the spatial
resolution is increased accordingly. Although there are many different spatial
discretization techniques available for Navier-Stokes flows, spectral methods
have long been known for providing unbeatable exponential convergence, provided
the domain is simple, and the boundary conditions are mathematically
well-posed.

Among the wide variety of two-dimensional lid-driven cavity flows, the one
that, for historical reasons, has received more attention is the one-sided
version \citep{kuhlmann2019}. This cavity consists of a square with one lid
sliding at a uniform speed, and the three other lids are kept stationary. In
this case, the flow is steady and stable even for high Reynolds numbers. The
first instability found is due to a Hopf bifurcation which may lie somewhere in
the interval [7500, 8100] \citep{kuhlmann2019} reported by different
studies. This uncertainty in the critical Reynolds number is due to multiple
factors, such as the large Reynolds number where the instability appears,
demands very high resolutions in space and time, as well as the different
spatial discretization techniques and methodologies used in the works to
identify the bifurcation (time integration or linear stability analysis). Due
to its simple geometry, the square lid-driven cavity has long been qualified as
a good candidate for a canonical benchmark problem. Still, lid-driven cavity
flow problems, in their original formulations, typically have singular boundary
conditions at the corners, where the velocity profile is discontinuous. In this
case, the exponential convergence of spectrally approximated flow solutions is
at stake. 

The corner-singularity problem has, however, been successfully circumvented
using different approaches, such as substraction methods \citep{botella1998},
that removes the leading terms of the asymptotic expansion of the solution of
the Navier-Stokes equations in the vicinity of the corners, thus recovering
spectral accuracy. Another successful strategy to preserve exponential
convergence when using spectral methods in lid-driven cavity flows is the
regularization of the boundary conditions by replacing the original
discontinuous profiles at the walls by polynomial or exponential velocity
distributions that mimic the original problem while preserving smoothness at
the corners \citep{shen1991, lopez2017}.

A more recent variant termed the four-sided cavity flow, has been proposed
\citep{wahba2009}. Here the four lids are moving at the same speed (top-bottom
and right-left lids moving rightwards-leftwards and upwards-downwards,
respectively). This cavity has the computational advantage of exhibiting a
variety of bifurcations at low or moderate Reynolds numbers. However, the
problem suffers from corner singularities due to the discontinuous boundary
conditions. This work addresses this issue by investigating a regularized
version of the four-sided cavity flow using a Chebyshev discretization
implemented in Julia, a high-performance programming language.

Julia, a free and open-source language for scientific computing, offers
performance comparable to compiled C/Fortran codes, making it an attractive
platform for implementing scientific computing. A developed Julia module
provides a reproducible example of the regularized cavity.

The regularized four-sided lid-driven cavity shows most of the primary
bifurcation scenarios. The flow undergoes instabilities, such as pitchfork,
saddle-node (fold), and Hopf. Predicting the precise location of the
bifurcations could present an amenable benchmark when testing and comparing
different discretization schemes and implementations. \\

This master project is structured as follows. First, the driven cavity flow
problem is reviewed. Then the governing equations are presented, followed by an
introduction to the regularized four-sided lid-driven cavity. The necessary is
developed in the theoretical part regarding spatial (spectral Chebyshev) and
time discretization. The numerical tools required for bifurcation analysis of
the flow are also explained. The subsequent step focuses on some implementation
details for the Julia programming language. Finally, the results of the
regularized flows are presented and discussed, including stable and unstable
solutions and the precise occurrences of bifurcations. The study concludes with
a summary of the objectives, limitations, and recommendations for future
research.
