%-------------------------------------------------------------------------------
%INTRODUCTION
%-------------------------------------------------------------------------------

\section{Introduction} \label{sec:intro}

Driven cavity flows have long served as essential benchmarks for validating
Navier-Stokes solvers. These problems can test spatial discretization
methodologies such as finite elements, finite differences, and spectral
methods. They may also be used to assess a variety of boundary condition
implementations and time-stepping schemes.

Current theoretical and computational research into Navier-Stokes flows goes
beyond the time integration of the equations of motion. Since the publication
of the seminal work by Eberhard Hopf 75 years ago \citep{hopf1948}, bifurcation
and dynamical systems theory have been vindicated as suitable deterministic
frameworks to understand hydrodynamic instabilities and the transition to
turbulent flow regimes.

Exploring the infinite-dimensional phase space of the Navier-Stokes equations
is necessary to anticipate instabilities and comprehend the emergence of new
flow states. That implies the seek for invariant sets or manifolds, the
simplest cases being equilibria (steady flows) and periodic orbits
(time-periodic flows). Although dynamically of high relevance, these manifolds
are often linearly unstable and, therefore, systematically overlooked by time
integrators of the Navier-Stokes equations, as time-stepping will only approach
local attractors.

Accurately computing steady or time-periodic flows, regardless of their
stability, requires the implementation of suitable continuation algorithms to
monitor them as the Reynolds number is varied \citep{kuznetsov2004}. Moreover,
the linear stability analysis of these invariant flows is crucial for
predicting potential local bifurcations in parameter space. However, this
analysis poses challenges due to the high-dimensional generalized eigenvalue
problem resulting from the local linearization of the Navier-Stokes operator.

Numerically reliable Navier-Stokes benchmarks require high accuracy and
robustness of its solutions. It is well known that some spatial numerical
discretizations of the Navier-Stokes equations may lead to spurious solutions
that typically appear for coarse grids, eventually vanishing when the spatial
resolution is increased accordingly. Although there are many different spatial
discretization techniques available for Navier-Stokes flows, spectral methods
have long been known for providing unbeatable exponential convergence, provided
the domain is simple, and the boundary conditions are mathematically
well-posed.

For historical reasons, the one-sided version has received more attention among
the wide variety of two-dimensional lid-driven cavity flows
\citep{kuhlmann2019}. This cavity consists of a square with one lid sliding at
a uniform speed, and the three other lids are kept stationary. In this case,
the flow is steady and stable even for high Reynolds numbers. The first
instability found is due to a Hopf bifurcation which may lie somewhere in the
interval [7500, 8100] reported by different studies \citep{kuhlmann2019}. This
uncertainty in the reported critical Reynolds number is due to multiple
factors. For instance,  the large Reynolds number where the instability appears
demands very high resolutions in space and time, as well as the different
spatial discretization techniques and methodologies used in the works to
identify the bifurcation (time integration or linear stability analysis). Due
to its simple geometry, the square lid-driven cavity has long been qualified as
a good candidate for a canonical benchmark problem. Still, lid-driven cavity
flow problems, in their original formulations, typically have singular boundary
conditions at the corners, where the velocity profile is discontinuous. In this
case, the exponential convergence of spectrally approximated flow solutions is
at stake. 

The corner-singularity problem has, however, been successfully circumvented
using different approaches, such as substraction methods \citep{botella1998},
which eliminates the dominant terms of the asymptotic expansion of the solution
to the Navier-Stokes equations near the corners, thus recovering spectral
accuracy. Another successful strategy to preserve exponential convergence when
using spectral methods in lid-driven cavity flows is the regularization of the
boundary conditions. By replacing the discontinuous profiles at the walls with
polynomial or exponential velocity distributions, the original problem can be
mimicked while preserving smoothness at the corners \citep{shen1991,
lopez2017}.

A more recent variant, termed the four-sided cavity flow, has been proposed
\citep{wahba2009}. Here the four lids are moving at the same speed (top-bottom
and right-left lids moving rightwards-leftwards and upwards-downwards,
respectively). This cavity has the computational advantage of exhibiting a
variety of bifurcations at low or moderate Reynolds numbers. However, the
problem suffers from corner singularities due to the discontinuous boundary
conditions. This work addresses this issue by investigating a regularized
version of the four-sided cavity flow using a Chebyshev discretization
implemented in Julia, a high-performance programming language.

Julia, a free and open-source language for scientific computing, offers
performance comparable to compiled C/Fortran codes, making it an attractive
platform for implementing scientific computing. A developed Julia module
provides a reproducible example of the regularized cavity.

The regularized four-sided lid-driven cavity shows most of the primary
bifurcation scenarios. The flow undergoes instabilities, such as pitchfork,
saddle-node (fold), and Hopf. Determining the precise location of the
bifurcations could present an amenable benchmark when testing and comparing
different discretization schemes and implementations. \\

This master project is structured as follows. First, a review of the driven
cavity flow problem is provided. Then the four-sided lid-driven cavity flow is
introduced, and the governing equations are presented, followed by an
introduction to the regularized version. The necessary theory is developed in
the theoretical part regarding spatial (spectral Chebyshev) and time
discretization. The numerical tools required for bifurcation analysis of the
flow are also explained. The subsequent step focuses on implementation details
for the Julia programming language. Finally, the results of the regularized
problem are presented and discussed, including stable and unstable solutions
and the precise occurrences of bifurcations. The study concludes with a summary
of the objectives, limitations, and recommendations for future research.
