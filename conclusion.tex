%-------------------------------------------------------------------------------
% CONCLUSION
%-------------------------------------------------------------------------------

\section{Conclusion} \label{concl}

In conclusion, this master project focused on the four-sided lid-driven cavity flow using
regularized boundary conditions. The study's spectral Chebyshev discretization
technique combined with the regularization yields highly accurate results
having an exponential convergence rate as the grid size increases. The
implementation was carried out using the programming language Julia and
provides an open-source example for reference. The regularized version of the
cavity flow shows a bifurcation diagram resembling the diagram with
discontinuous boundary conditions. A super-critical Hopf bifurcation was found
at a Reynolds number of $348.319$. Additionally, a pitchfork bifurcation occurs
just prior to the saddle-node bifurcation at around $353.347$ Reynolds. This
connected unstable branch breaks the $\pi$-rotation symmetry of the asymmetric
solutions. On the other hand, for the saddle-node bifurcation, further
investigation is required due to an unclear eigenvalue crossing, possibly
associated with a complex conjugate pair. The presence of three pitchfork
bifurcations and the Hopf bifurcation in this problem offers robust scenarios
to benchmark a Navier-Stokes solver. The regularized four-sided lid-driven
cavity only requires slight adaptations from the classical one-sided version,
making it easily applicable to existing codes. This modified setup allows for
testing the multiplicity of states and periodic orbits at low to moderate
Reynolds numbers, providing the means to assess a solver's capability in
identifying these solution types and the oscillatory behavior. In general, it
can make up for an interesting bifurcation benchmark.
